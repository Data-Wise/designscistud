\documentclass[]{article}
\usepackage{lmodern}
\usepackage{amssymb,amsmath}
\usepackage{ifxetex,ifluatex}
\usepackage{fixltx2e} % provides \textsubscript
\ifnum 0\ifxetex 1\fi\ifluatex 1\fi=0 % if pdftex
  \usepackage[T1]{fontenc}
  \usepackage[utf8]{inputenc}
\else % if luatex or xelatex
  \ifxetex
    \usepackage{mathspec}
  \else
    \usepackage{fontspec}
  \fi
  \defaultfontfeatures{Ligatures=TeX,Scale=MatchLowercase}
\fi
% use upquote if available, for straight quotes in verbatim environments
\IfFileExists{upquote.sty}{\usepackage{upquote}}{}
% use microtype if available
\IfFileExists{microtype.sty}{%
\usepackage{microtype}
\UseMicrotypeSet[protrusion]{basicmath} % disable protrusion for tt fonts
}{}
\usepackage[margin=1in]{geometry}
\usepackage{hyperref}
\hypersetup{unicode=true,
            pdftitle={STA305/1004 - Homework \#4},
            pdfborder={0 0 0},
            breaklinks=true}
\urlstyle{same}  % don't use monospace font for urls
\usepackage{graphicx,grffile}
\makeatletter
\def\maxwidth{\ifdim\Gin@nat@width>\linewidth\linewidth\else\Gin@nat@width\fi}
\def\maxheight{\ifdim\Gin@nat@height>\textheight\textheight\else\Gin@nat@height\fi}
\makeatother
% Scale images if necessary, so that they will not overflow the page
% margins by default, and it is still possible to overwrite the defaults
% using explicit options in \includegraphics[width, height, ...]{}
\setkeys{Gin}{width=\maxwidth,height=\maxheight,keepaspectratio}
\IfFileExists{parskip.sty}{%
\usepackage{parskip}
}{% else
\setlength{\parindent}{0pt}
\setlength{\parskip}{6pt plus 2pt minus 1pt}
}
\setlength{\emergencystretch}{3em}  % prevent overfull lines
\providecommand{\tightlist}{%
  \setlength{\itemsep}{0pt}\setlength{\parskip}{0pt}}
\setcounter{secnumdepth}{0}
% Redefines (sub)paragraphs to behave more like sections
\ifx\paragraph\undefined\else
\let\oldparagraph\paragraph
\renewcommand{\paragraph}[1]{\oldparagraph{#1}\mbox{}}
\fi
\ifx\subparagraph\undefined\else
\let\oldsubparagraph\subparagraph
\renewcommand{\subparagraph}[1]{\oldsubparagraph{#1}\mbox{}}
\fi

%%% Use protect on footnotes to avoid problems with footnotes in titles
\let\rmarkdownfootnote\footnote%
\def\footnote{\protect\rmarkdownfootnote}

%%% Change title format to be more compact
\usepackage{titling}

% Create subtitle command for use in maketitle
\newcommand{\subtitle}[1]{
  \posttitle{
    \begin{center}\large#1\end{center}
    }
}

\setlength{\droptitle}{-2em}
  \title{STA305/1004 - Homework \#4}
  \pretitle{\vspace{\droptitle}\centering\huge}
  \posttitle{\par}
  \author{}
  \preauthor{}\postauthor{}
  \predate{\centering\large\emph}
  \postdate{\par}
  \date{Due Date: April 5, 2017}


\begin{document}
\maketitle

\textbf{Due date:} Electronic submission via Crowdmark by Wednesday,
April 5, 2017 at 22:00. NB: e-mail submissions will NOT be accepted. DO
NOT SUBMIT MULTIPLE DOCUMENTS (E.G., ONE DOCUMENT WITH A DESCRIPTION AND
ANOTHER WITH GRAPHS)

Each student (no groups allowed) will plan and perform a homemade
factorial experiment. This includes collecting and analyzing the data in
R. I leave it to each individual student to decide what he or she wants
to study. The number of possible topics is very large. It's very
important you pick a topic that you are interested in and will enjoy
working on.

STA1004 students: If you are working on a research project where a
factorial design can be implemented then I encourage you to use this
project as your topic.

The report should not be longer than 4 pages including tables and plots
and be \textbf{submitted in one document}. An example report can be
found on portal. Feel free to use this report as a template.

The report should include the following three sections:

\section{Description of the design. (One page
maximum)}\label{description-of-the-design.-one-page-maximum}

The design should be a replicated or unreplicated full or fractional
factorial experiment. Include details on how and why you conducted the
experiment. What do you hope to learn by doing this experiment? You may
want to include a small picture of your experimental apparatus if it
will help your description.

Grade:

5 -- Excellent: Strong evidence of original thinking and a clear
explanation of how and why they conducted the experiment.

4- Good: Grasped the basics of designing a factorial study; a good
explanation of how and why they did the experiment.

3- Adequate: Understood the basics of designing a factorial study, but
may not have designed a factorial experiment. Provided an adequate
explanation of the design.

2- Marginal: Some evidence of understanding the basic design of a
factorial study. Provided a poor explanation of their design.

1- Inadequate: Little evidence of even a superficial understanding of a
factorial design. Little explanation about how or why the design was
chosen.

\section{Analysis of the data. (Two pages
maximum)}\label{analysis-of-the-data.-two-pages-maximum}

Include appropriate plots and calculations such as: main effects and
interactions; estimated variance of the effect (if replicated);
confidence intervals for true values of effects (if replicated); Lenth
plot; or half normal plot.

Grade:

5 -- Excellent: Strong evidence of data analysis skills. Probably used R
to do calculations and plots, but calculations and plots might also be
done neatly by hand.

4- Good: Good evidence of data analysis skills. Appropriate calculations
were done, and maybe appropriate plots were included.

3- Adequate: Understood the basics of required data analysis.

2- Marginal: Some evidence of understanding the basic data analysis
required, but might not have carried out all the appropriate
calculations and plots.

1- Inadequate: Little evidence of even a superficial understanding of
the data analysis required to analyse a factorial design.

\section{Conclusions. (One page
maximum)}\label{conclusions.-one-page-maximum}

What conclusions can you make based on the results of your experiment?
Write a paragraph or two outlining these conclusions.

Grade:

5 -- Excellent: Conclusions are highly appropriate given the experiment
conducted. Clearly written.

4- Good: Conclusions are appropriate given the experimental context.
Writing is good.

3- Adequate: Some conclusions are appropriate; other obvious conclusions
might be missing.

2- Marginal: Some evidence that there was an understanding of the basic
conclusions, but several obvious conclusions not stated.

1- Inadequate: Little evidence of even a superficial understanding of
the conclusions that can be drawn from the experiment.


\end{document}
